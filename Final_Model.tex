\documentclass[]{article}
\usepackage{lmodern}
\usepackage{amssymb,amsmath}
\usepackage{ifxetex,ifluatex}
\usepackage{fixltx2e} % provides \textsubscript
\ifnum 0\ifxetex 1\fi\ifluatex 1\fi=0 % if pdftex
  \usepackage[T1]{fontenc}
  \usepackage[utf8]{inputenc}
\else % if luatex or xelatex
  \ifxetex
    \usepackage{mathspec}
  \else
    \usepackage{fontspec}
  \fi
  \defaultfontfeatures{Ligatures=TeX,Scale=MatchLowercase}
\fi
% use upquote if available, for straight quotes in verbatim environments
\IfFileExists{upquote.sty}{\usepackage{upquote}}{}
% use microtype if available
\IfFileExists{microtype.sty}{%
\usepackage[]{microtype}
\UseMicrotypeSet[protrusion]{basicmath} % disable protrusion for tt fonts
}{}
\PassOptionsToPackage{hyphens}{url} % url is loaded by hyperref
\usepackage[unicode=true]{hyperref}
\hypersetup{
            pdftitle={Final\_Model},
            pdfauthor={Tyler Cobian, Jeremy Knox, Alex Smith},
            pdfborder={0 0 0},
            breaklinks=true}
\urlstyle{same}  % don't use monospace font for urls
\usepackage[margin=1in]{geometry}
\usepackage{color}
\usepackage{fancyvrb}
\newcommand{\VerbBar}{|}
\newcommand{\VERB}{\Verb[commandchars=\\\{\}]}
\DefineVerbatimEnvironment{Highlighting}{Verbatim}{commandchars=\\\{\}}
% Add ',fontsize=\small' for more characters per line
\usepackage{framed}
\definecolor{shadecolor}{RGB}{248,248,248}
\newenvironment{Shaded}{\begin{snugshade}}{\end{snugshade}}
\newcommand{\KeywordTok}[1]{\textcolor[rgb]{0.13,0.29,0.53}{\textbf{#1}}}
\newcommand{\DataTypeTok}[1]{\textcolor[rgb]{0.13,0.29,0.53}{#1}}
\newcommand{\DecValTok}[1]{\textcolor[rgb]{0.00,0.00,0.81}{#1}}
\newcommand{\BaseNTok}[1]{\textcolor[rgb]{0.00,0.00,0.81}{#1}}
\newcommand{\FloatTok}[1]{\textcolor[rgb]{0.00,0.00,0.81}{#1}}
\newcommand{\ConstantTok}[1]{\textcolor[rgb]{0.00,0.00,0.00}{#1}}
\newcommand{\CharTok}[1]{\textcolor[rgb]{0.31,0.60,0.02}{#1}}
\newcommand{\SpecialCharTok}[1]{\textcolor[rgb]{0.00,0.00,0.00}{#1}}
\newcommand{\StringTok}[1]{\textcolor[rgb]{0.31,0.60,0.02}{#1}}
\newcommand{\VerbatimStringTok}[1]{\textcolor[rgb]{0.31,0.60,0.02}{#1}}
\newcommand{\SpecialStringTok}[1]{\textcolor[rgb]{0.31,0.60,0.02}{#1}}
\newcommand{\ImportTok}[1]{#1}
\newcommand{\CommentTok}[1]{\textcolor[rgb]{0.56,0.35,0.01}{\textit{#1}}}
\newcommand{\DocumentationTok}[1]{\textcolor[rgb]{0.56,0.35,0.01}{\textbf{\textit{#1}}}}
\newcommand{\AnnotationTok}[1]{\textcolor[rgb]{0.56,0.35,0.01}{\textbf{\textit{#1}}}}
\newcommand{\CommentVarTok}[1]{\textcolor[rgb]{0.56,0.35,0.01}{\textbf{\textit{#1}}}}
\newcommand{\OtherTok}[1]{\textcolor[rgb]{0.56,0.35,0.01}{#1}}
\newcommand{\FunctionTok}[1]{\textcolor[rgb]{0.00,0.00,0.00}{#1}}
\newcommand{\VariableTok}[1]{\textcolor[rgb]{0.00,0.00,0.00}{#1}}
\newcommand{\ControlFlowTok}[1]{\textcolor[rgb]{0.13,0.29,0.53}{\textbf{#1}}}
\newcommand{\OperatorTok}[1]{\textcolor[rgb]{0.81,0.36,0.00}{\textbf{#1}}}
\newcommand{\BuiltInTok}[1]{#1}
\newcommand{\ExtensionTok}[1]{#1}
\newcommand{\PreprocessorTok}[1]{\textcolor[rgb]{0.56,0.35,0.01}{\textit{#1}}}
\newcommand{\AttributeTok}[1]{\textcolor[rgb]{0.77,0.63,0.00}{#1}}
\newcommand{\RegionMarkerTok}[1]{#1}
\newcommand{\InformationTok}[1]{\textcolor[rgb]{0.56,0.35,0.01}{\textbf{\textit{#1}}}}
\newcommand{\WarningTok}[1]{\textcolor[rgb]{0.56,0.35,0.01}{\textbf{\textit{#1}}}}
\newcommand{\AlertTok}[1]{\textcolor[rgb]{0.94,0.16,0.16}{#1}}
\newcommand{\ErrorTok}[1]{\textcolor[rgb]{0.64,0.00,0.00}{\textbf{#1}}}
\newcommand{\NormalTok}[1]{#1}
\usepackage{graphicx,grffile}
\makeatletter
\def\maxwidth{\ifdim\Gin@nat@width>\linewidth\linewidth\else\Gin@nat@width\fi}
\def\maxheight{\ifdim\Gin@nat@height>\textheight\textheight\else\Gin@nat@height\fi}
\makeatother
% Scale images if necessary, so that they will not overflow the page
% margins by default, and it is still possible to overwrite the defaults
% using explicit options in \includegraphics[width, height, ...]{}
\setkeys{Gin}{width=\maxwidth,height=\maxheight,keepaspectratio}
\IfFileExists{parskip.sty}{%
\usepackage{parskip}
}{% else
\setlength{\parindent}{0pt}
\setlength{\parskip}{6pt plus 2pt minus 1pt}
}
\setlength{\emergencystretch}{3em}  % prevent overfull lines
\providecommand{\tightlist}{%
  \setlength{\itemsep}{0pt}\setlength{\parskip}{0pt}}
\setcounter{secnumdepth}{0}
% Redefines (sub)paragraphs to behave more like sections
\ifx\paragraph\undefined\else
\let\oldparagraph\paragraph
\renewcommand{\paragraph}[1]{\oldparagraph{#1}\mbox{}}
\fi
\ifx\subparagraph\undefined\else
\let\oldsubparagraph\subparagraph
\renewcommand{\subparagraph}[1]{\oldsubparagraph{#1}\mbox{}}
\fi

% set default figure placement to htbp
\makeatletter
\def\fps@figure{htbp}
\makeatother


\title{Final\_Model}
\author{Tyler Cobian, Jeremy Knox, Alex Smith}
\date{5/26/2020}

\begin{document}
\maketitle

Load the packages

\begin{Shaded}
\begin{Highlighting}[]
\KeywordTok{library}\NormalTok{(tidyverse)}
\end{Highlighting}
\end{Shaded}

\begin{verbatim}
## -- Attaching packages --------------------------------------- tidyverse 1.3.0 --
\end{verbatim}

\begin{verbatim}
## v ggplot2 3.3.0     v purrr   0.3.4
## v tibble  3.0.1     v dplyr   0.8.5
## v tidyr   1.0.2     v stringr 1.4.0
## v readr   1.3.1     v forcats 0.5.0
\end{verbatim}

\begin{verbatim}
## -- Conflicts ------------------------------------------ tidyverse_conflicts() --
## x dplyr::filter() masks stats::filter()
## x dplyr::lag()    masks stats::lag()
\end{verbatim}

\begin{Shaded}
\begin{Highlighting}[]
\KeywordTok{library}\NormalTok{(sensitivity)}
\end{Highlighting}
\end{Shaded}

\begin{verbatim}
## Registered S3 method overwritten by 'sensitivity':
##   method    from 
##   print.src dplyr
\end{verbatim}

\begin{verbatim}
## 
## Attaching package: 'sensitivity'
\end{verbatim}

\begin{verbatim}
## The following object is masked from 'package:dplyr':
## 
##     src
\end{verbatim}

\begin{Shaded}
\begin{Highlighting}[]
\KeywordTok{library}\NormalTok{(deSolve)}
\KeywordTok{library}\NormalTok{(ggplot2)}
\end{Highlighting}
\end{Shaded}

\begin{Shaded}
\begin{Highlighting}[]
\NormalTok{SEIR <-}\StringTok{ }\ControlFlowTok{function}\NormalTok{(time, state, parms) \{}
  \CommentTok{# SEIR model with social distancing of rho}
  \CommentTok{# Model adapted from https://cran.r-project.org/web/packages/shinySIR/vignettes/Vignette.html and}
  \CommentTok{# https://towardsdatascience.com/social-distancing-to-slow-the-coronavirus-768292f04296}
  \CommentTok{# PARMS:}
  \CommentTok{# α is the inverse of the incubation period (1/t_incubation)}
  \CommentTok{# β is the average contact rate in the population}
  \CommentTok{# γ is the inverse of the mean infectious period (1/t_infectious)}
  \CommentTok{# ρ rho is the social distancing effect, 0 being everyone locked down}
  \CommentTok{# VARIABLES:}
  \CommentTok{# S = SUSCEPTIBLE, E = EXPOSED, I = INFECTED, R = RECOVERED}
  \CommentTok{# }
    \KeywordTok{with}\NormalTok{(}\KeywordTok{as.list}\NormalTok{(}\KeywordTok{c}\NormalTok{(state, parms, }\DataTypeTok{i =}\NormalTok{ time)),\{}
      
        \CommentTok{# Change in SUSCEPTIBLE}
\NormalTok{        dS <-}\StringTok{ }\OperatorTok{-}\StringTok{ }\NormalTok{rho[i] }\OperatorTok{*}\StringTok{ }\NormalTok{beta }\OperatorTok{*}\StringTok{ }\NormalTok{S }\OperatorTok{*}\StringTok{ }\NormalTok{I }
        
        \CommentTok{# Change in EXPOSED}
\NormalTok{        dE <-}\StringTok{ }\NormalTok{rho[i] }\OperatorTok{*}\StringTok{ }\NormalTok{beta }\OperatorTok{*}\StringTok{ }\NormalTok{S }\OperatorTok{*}\StringTok{ }\NormalTok{I }\OperatorTok{-}\StringTok{ }\NormalTok{alpha }\OperatorTok{*}\StringTok{ }\NormalTok{E}
        
        \CommentTok{# Change in INFECTED}
\NormalTok{        dI <-}\StringTok{ }\NormalTok{alpha }\OperatorTok{*}\StringTok{ }\NormalTok{E }\OperatorTok{-}\StringTok{ }\NormalTok{gamma }\OperatorTok{*}\StringTok{ }\NormalTok{I}
        
        \CommentTok{# Change in RECOVERED}
\NormalTok{        dR <-}\StringTok{ }\NormalTok{gamma }\OperatorTok{*}\StringTok{ }\NormalTok{I }
        
    \KeywordTok{return}\NormalTok{(}\KeywordTok{list}\NormalTok{(}\KeywordTok{c}\NormalTok{(dS, dE, dI, dR, rho)))}
\NormalTok{    \})}
\NormalTok{\}}
\CommentTok{# Set parameteres and intial conditions   ***THESE CONDITIONS ARE VERY SENSITIVE AND UNKNOWN IN REALITY***}
\NormalTok{time <-}\StringTok{ }\KeywordTok{seq}\NormalTok{(}\DecValTok{1}\NormalTok{, }\DecValTok{180}\NormalTok{, }\DataTypeTok{by =} \DecValTok{1}\NormalTok{)}

\NormalTok{rho_start =}\StringTok{ }\FloatTok{0.6}
\NormalTok{rho_end =}\StringTok{ }\DecValTok{1}
\NormalTok{rho_diff =}\StringTok{ }\KeywordTok{abs}\NormalTok{(rho_end }\OperatorTok{-}\StringTok{ }\NormalTok{rho_start)}
\NormalTok{rho =}\StringTok{ }\KeywordTok{seq}\NormalTok{(rho_start, rho_end, }\DataTypeTok{by =}\NormalTok{ rho_diff}\OperatorTok{/}\KeywordTok{length}\NormalTok{(time))}

\NormalTok{init <-}\StringTok{ }\KeywordTok{c}\NormalTok{(}\DataTypeTok{S =}\DecValTok{1}\OperatorTok{-}\NormalTok{.}\DecValTok{001}\NormalTok{, }\DataTypeTok{E =}\NormalTok{.}\DecValTok{001}\NormalTok{, }\DataTypeTok{I =} \DecValTok{0}\NormalTok{, }\DataTypeTok{R =} \DecValTok{0}\NormalTok{, rho) }\CommentTok{# assuming 0.1% of the population has been infected}
\NormalTok{parms <-}\StringTok{ }\KeywordTok{c}\NormalTok{(}\DataTypeTok{alpha =} \FloatTok{0.2}\NormalTok{, }\DataTypeTok{beta =} \DecValTok{2}\NormalTok{, }\DataTypeTok{gamma =} \DecValTok{1}\NormalTok{) }\CommentTok{# assuming a stay at home order that reduces social contact by 20%}
\CommentTok{# }\AlertTok{NOTE}\CommentTok{: Let us define R0 = betta/gamma, which is how fast the disease spreads}
\CommentTok{# When R0 > 1 the disease will grow exponentially, likewise R0 < 1 dies exponentially}

\CommentTok{# Solve using ODE}
\NormalTok{out <-}\StringTok{ }\KeywordTok{ode}\NormalTok{(}\DataTypeTok{y =}\NormalTok{ init, }\DataTypeTok{times =}\NormalTok{ time, }\DataTypeTok{func =}\NormalTok{ SEIR, }\DataTypeTok{parms =}\NormalTok{ parms)}

\CommentTok{# Change to data frame}
\NormalTok{out <-}\StringTok{ }\KeywordTok{as.data.frame}\NormalTok{(out)}
\NormalTok{out}\OperatorTok{$}\NormalTok{time <-}\StringTok{ }\NormalTok{time }\CommentTok{# Delete time variable}
\CommentTok{# }
\CommentTok{# ggplot(data=out, aes(x = time)) +}
\CommentTok{#   geom_point(aes(y=rho, color="Rho"))}


\NormalTok{out_clean<-}\StringTok{ }\NormalTok{out }\OperatorTok\StringTok{ }
\StringTok{  }\KeywordTok{select}\NormalTok{(time, S, E, I, R, V6)}


\CommentTok{#Plot results}
\KeywordTok{ggplot}\NormalTok{(}\DataTypeTok{data=}\NormalTok{out_clean, }\KeywordTok{aes}\NormalTok{(}\DataTypeTok{x =}\NormalTok{ time)) }\OperatorTok{+}
\StringTok{  }\KeywordTok{geom_point}\NormalTok{(}\KeywordTok{aes}\NormalTok{(}\DataTypeTok{y=}\NormalTok{S, }\DataTypeTok{color=}\StringTok{"Susceptible"}\NormalTok{)) }\OperatorTok{+}
\StringTok{  }\KeywordTok{geom_point}\NormalTok{(}\KeywordTok{aes}\NormalTok{(}\DataTypeTok{y=}\NormalTok{E, }\DataTypeTok{color=}\StringTok{"Exposed"}\NormalTok{)) }\OperatorTok{+}
\StringTok{  }\KeywordTok{geom_point}\NormalTok{(}\KeywordTok{aes}\NormalTok{(}\DataTypeTok{y=}\NormalTok{I, }\DataTypeTok{color=}\StringTok{"Infected"}\NormalTok{)) }\OperatorTok{+}
\StringTok{  }\KeywordTok{geom_point}\NormalTok{(}\KeywordTok{aes}\NormalTok{(}\DataTypeTok{y=}\NormalTok{R, }\DataTypeTok{color=}\StringTok{"Recovered (Immune)"}\NormalTok{)) }\OperatorTok{+}
\StringTok{  }\CommentTok{# scale_colour_manual(name=) +}
\StringTok{  }\KeywordTok{theme_minimal}\NormalTok{() }\OperatorTok{+}
\StringTok{  }\KeywordTok{labs}\NormalTok{(}\DataTypeTok{x =} \StringTok{"Days"}\NormalTok{, }\DataTypeTok{y =} \StringTok{"Percent of Population"}\NormalTok{, }\DataTypeTok{color =} \StringTok{""}\NormalTok{)}
\end{Highlighting}
\end{Shaded}

\includegraphics{Final_Model_files/figure-latex/SEIR Model-1.pdf}

Make data data set that accounts for

\begin{Shaded}
\begin{Highlighting}[]
\NormalTok{mean_temp<-}\StringTok{ }\KeywordTok{rnorm}\NormalTok{(}\DataTypeTok{n =} \DecValTok{180}\NormalTok{, }\DataTypeTok{mean =} \DecValTok{65}\NormalTok{, }\DataTypeTok{sd =} \DecValTok{5}\NormalTok{) }\CommentTok{# normal distribution of temperature}
\NormalTok{avg_mile_day =}\StringTok{ }\DecValTok{974219178} \CommentTok{# averge miles driven a day in CA noramlly}
\NormalTok{weather_coeff =}\StringTok{ }\DecValTok{1000000} \CommentTok{# coefficient of weather on averge miles driven a day in CA}
\NormalTok{shelter_coeff =}\StringTok{ }\DecValTok{1000000} \CommentTok{# coefficient of shelter strength on miles driven a day in CA}
\NormalTok{mile_emissions =}\StringTok{ }\FloatTok{0.411} \CommentTok{# Emissions factor of 1 mile driven }
\CommentTok{# add mean temperature into the covid data }
\NormalTok{out_clean}\OperatorTok{$}\NormalTok{temp<-}\StringTok{ }\NormalTok{mean_temp}
\CommentTok{# Put in a for loop for miles driven as dependant on infections_shelter and temperature}

\NormalTok{out_clean}\OperatorTok{$}\NormalTok{miles_driven<-}\StringTok{ }\KeywordTok{rep}\NormalTok{(}\OtherTok{NA}\NormalTok{, }\DecValTok{180}\NormalTok{)}

\ControlFlowTok{for}\NormalTok{ (i }\ControlFlowTok{in} \KeywordTok{length}\NormalTok{(out_clean}\OperatorTok{$}\NormalTok{time))\{}
\NormalTok{  out_clean}\OperatorTok{$}\NormalTok{miles_driven =}\StringTok{ }\NormalTok{avg_mile_day }\OperatorTok{+}\StringTok{ }\NormalTok{weather_coeff }\OperatorTok{*}\StringTok{ }\NormalTok{(out_clean}\OperatorTok{$}\NormalTok{temp }\OperatorTok{-}\StringTok{ }\KeywordTok{mean}\NormalTok{(out_clean}\OperatorTok{$}\NormalTok{temp)) }\OperatorTok{+}\StringTok{ }\NormalTok{shelter_coeff }\OperatorTok{*}\StringTok{ }\NormalTok{(out_clean}\OperatorTok{$}\NormalTok{V6)}
\NormalTok{\}}
\CommentTok{# loop for ghg emissions}
\ControlFlowTok{for}\NormalTok{ (i }\ControlFlowTok{in} \KeywordTok{length}\NormalTok{(out_clean}\OperatorTok{$}\NormalTok{time))\{}
\NormalTok{  out_clean}\OperatorTok{$}\NormalTok{ghg =}\StringTok{ }\NormalTok{out_clean}\OperatorTok{$}\NormalTok{miles_driven }\OperatorTok{*}\StringTok{ }\NormalTok{mile_emissions}
\NormalTok{\}}
\end{Highlighting}
\end{Shaded}

Vizuals for GHG emissiona and driving

\begin{Shaded}
\begin{Highlighting}[]
\KeywordTok{options}\NormalTok{(}\DataTypeTok{scipen =} \DecValTok{99}\NormalTok{)}

\CommentTok{# plot for driving}
\NormalTok{drive<-}\StringTok{ }\KeywordTok{ggplot}\NormalTok{(out_clean, }\KeywordTok{aes}\NormalTok{(}\DataTypeTok{x =}\NormalTok{ time, }\DataTypeTok{y =}\NormalTok{ ghg))}\OperatorTok{+}
\StringTok{  }\KeywordTok{geom_line}\NormalTok{(}\DataTypeTok{color =} \StringTok{"darkorange4"}\NormalTok{)}\OperatorTok{+}
\StringTok{  }\KeywordTok{theme_minimal}\NormalTok{()}\OperatorTok{+}
\StringTok{  }\KeywordTok{labs}\NormalTok{(}\DataTypeTok{x =} \StringTok{"Time"}\NormalTok{, }\DataTypeTok{y =} \StringTok{"kgCO2e"}\NormalTok{, }\DataTypeTok{title =} \StringTok{"Mobile GHG Emission"}\NormalTok{, }\DataTypeTok{subtitle =} \StringTok{"Impact of weather and shelter in place measures"}\NormalTok{)}
\NormalTok{drive}
\end{Highlighting}
\end{Shaded}

\includegraphics{Final_Model_files/figure-latex/unnamed-chunk-3-1.pdf}

Sobol sensativity on gamma and beta

\begin{Shaded}
\begin{Highlighting}[]
\NormalTok{ghg =}\StringTok{ }\ControlFlowTok{function}\NormalTok{(avg_mile_day, weather_coeff, shelter_coeff, mean_temp, V6)\{}
\NormalTok{  miles_driven =}\StringTok{ }\NormalTok{avg_mile_day }\OperatorTok{+}\StringTok{ }\NormalTok{weather_coeff }\OperatorTok{*}\StringTok{ }\NormalTok{mean_temp }\OperatorTok{-}\StringTok{ }\KeywordTok{mean}\NormalTok{(mean_temp) }\OperatorTok{+}\StringTok{ }\NormalTok{shelter_coeff }\OperatorTok{*}\StringTok{ }\NormalTok{V6}
\NormalTok{  ghg =}\StringTok{ }\NormalTok{miles_driven }\OperatorTok{*}\StringTok{ }\NormalTok{mile_emissions}
  
  \KeywordTok{return}\NormalTok{(}\KeywordTok{list}\NormalTok{(ghg))}
\NormalTok{\}}


\CommentTok{# establish the number of runs}
\NormalTok{np =}\StringTok{ }\DecValTok{1000}

\CommentTok{# make one distribution }
\NormalTok{weather_coeff<-}\StringTok{ }\KeywordTok{rnorm}\NormalTok{(}\DataTypeTok{mean =} \DecValTok{1000}\NormalTok{, }\DataTypeTok{sd =} \DecValTok{300}\NormalTok{, }\DataTypeTok{n =}\NormalTok{ np)}
\NormalTok{shelter_coeff<-}\StringTok{ }\KeywordTok{rnorm}\NormalTok{(}\DataTypeTok{mean =} \DecValTok{1000000}\NormalTok{, }\DataTypeTok{sd =} \DecValTok{300}\NormalTok{, }\DataTypeTok{n =}\NormalTok{ np)}

\CommentTok{# make a data frame of this gamma}
\NormalTok{X1<-}\StringTok{ }\KeywordTok{cbind.data.frame}\NormalTok{(weather_coeff, shelter_coeff)}

\CommentTok{# make a second data set}
\NormalTok{weather_coeff<-}\StringTok{ }\KeywordTok{rnorm}\NormalTok{(}\DataTypeTok{mean =} \DecValTok{1000}\NormalTok{, }\DataTypeTok{sd =} \DecValTok{300}\NormalTok{, }\DataTypeTok{n =}\NormalTok{ np)}
\NormalTok{shelter_coeff<-}\StringTok{ }\KeywordTok{rnorm}\NormalTok{(}\DataTypeTok{mean =} \DecValTok{1000000}\NormalTok{, }\DataTypeTok{sd =} \DecValTok{300}\NormalTok{, }\DataTypeTok{n =}\NormalTok{ np)}

\CommentTok{# make a data frame of this gamma}
\NormalTok{X2<-}\StringTok{ }\KeywordTok{cbind.data.frame}\NormalTok{(weather_coeff, shelter_coeff)}

\CommentTok{# apply sobol}
\NormalTok{covid_sens<-}\StringTok{ }\KeywordTok{sobol2007}\NormalTok{(}\DataTypeTok{model =} \OtherTok{NULL}\NormalTok{, X1, X2, }\DataTypeTok{nboot =} \DecValTok{100}\NormalTok{)}

\CommentTok{# avg_mile_day = 974219178}
\NormalTok{mean_temp<-}\StringTok{ }\KeywordTok{rnorm}\NormalTok{(}\DataTypeTok{n =} \DecValTok{180}\NormalTok{, }\DataTypeTok{mean =} \DecValTok{65}\NormalTok{, }\DataTypeTok{sd =} \DecValTok{5}\NormalTok{) }



\CommentTok{# Run model for all parameter sets}
\NormalTok{res =}\StringTok{ }\KeywordTok{mapply}\NormalTok{(}\DataTypeTok{FUN =}\NormalTok{ ghg,}
             \DataTypeTok{weather_coeff =}\NormalTok{ covid_sens}\OperatorTok{$}\NormalTok{X}\OperatorTok{$}\NormalTok{weather_coeff,}
             \DataTypeTok{shelter_coeff =}\NormalTok{ covid_sens}\OperatorTok{$}\NormalTok{X}\OperatorTok{$}\NormalTok{shelter_coeff,}
             \DataTypeTok{MoreArgs =} \KeywordTok{list}\NormalTok{(}\DataTypeTok{avg_mile_day =} \DecValTok{974219178}\NormalTok{,}
                             \DataTypeTok{mean_temp =} \DecValTok{65}\NormalTok{,}
                             \DataTypeTok{V6 =} \FloatTok{0.7}
\NormalTok{                            ))}

\CommentTok{# Unlist the results from res}
\NormalTok{res<-}\StringTok{ }\KeywordTok{unlist}\NormalTok{(res)}

\NormalTok{ghg_sobel =}\StringTok{ }\NormalTok{sensitivity}\OperatorTok{::}\KeywordTok{tell}\NormalTok{(covid_sens, res)}

\CommentTok{# first-order indices }
\NormalTok{ghg_sobel}\OperatorTok{$}\NormalTok{S}
\end{Highlighting}
\end{Shaded}

\begin{verbatim}
##                original       bias std. error   min. c.i.  max. c.i.
## weather_coeff  93.36256 132.705376 2477.22445 -5421.37841 5182.57068
## shelter_coeff -31.86608   2.953257   23.68423   -86.27494   18.98274
\end{verbatim}

\begin{Shaded}
\begin{Highlighting}[]
\CommentTok{# total sensitivity index }
\NormalTok{ghg_sobel}\OperatorTok{$}\NormalTok{T}
\end{Highlighting}
\end{Shaded}

\begin{verbatim}
##                original        bias std. error   min. c.i.  max. c.i.
## weather_coeff -91.18673 -132.714996 2477.22793 -5180.35752 5423.64548
## shelter_coeff  31.86628   -2.953258   23.68432   -18.98265   86.27535
\end{verbatim}

\begin{Shaded}
\begin{Highlighting}[]
\KeywordTok{print}\NormalTok{(ghg_sobel)}
\end{Highlighting}
\end{Shaded}

\begin{verbatim}
## 
## Call:
## sobol2007(model = NULL, X1 = X1, X2 = X2, nboot = 100)
## 
## Model runs: 4000 
## 
## First order indices:
##                original       bias std. error   min. c.i.  max. c.i.
## weather_coeff  93.36256 132.705376 2477.22445 -5421.37841 5182.57068
## shelter_coeff -31.86608   2.953257   23.68423   -86.27494   18.98274
## 
## Total indices:
##                original        bias std. error   min. c.i.  max. c.i.
## weather_coeff -91.18673 -132.714996 2477.22793 -5180.35752 5423.64548
## shelter_coeff  31.86628   -2.953258   23.68432   -18.98265   86.27535
\end{verbatim}

\begin{Shaded}
\begin{Highlighting}[]
\KeywordTok{plot}\NormalTok{(ghg_sobel)}
\end{Highlighting}
\end{Shaded}

\includegraphics{Final_Model_files/figure-latex/unnamed-chunk-4-1.pdf}

\begin{Shaded}
\begin{Highlighting}[]
\CommentTok{# make a data frame for plotting}
\NormalTok{both =}\StringTok{ }\KeywordTok{cbind.data.frame}\NormalTok{(ghg_sobel}\OperatorTok{$}\NormalTok{X, }\DataTypeTok{ghg=}\NormalTok{ghg_sobel}\OperatorTok{$}\NormalTok{y)}
\end{Highlighting}
\end{Shaded}

\end{document}
